\section{Introduction}
{\label{intro}}
Graph theory is a combination of mathematics and computer science. It has been developed for a long time ago and many applications in this field are well developed. One of the topics is the graph coloring which is applied to many fields such as education, science and technology, etc.\\\\
Depending on the applications, it is flexible to choose the appropriate graph coloring algorithms to apply. However, the number of colors which applies to the projects needs to be considered. Vizing's theorem indicated how to use the number of colors to color the edges of a valid graph.\\\\
Our project is mainly based on the framework ``Graph" from the supervisor. We applied the core of framework and the improvement of it to display on the graphical user interface (GUI). Furthermore, our program is a visualization for a coloring graph theorem, which is based on the paper of J. Misra and David Gries: ``A constructive proof of Vizing's Theorem". It to simplify the algorithm by B{\'e}la Bollob{\'a}s$^{\cite{graphtheory}}$. It displays the result on GUI. The functionalities are built as following:
\begin{itemize}
\item Draw the graph: we display it into the GUI.
\item Implement the algorithm Breadth-first search (BFS): From the previous step, we implement the Breadth-first search algorithm$^{\cite{even2011graph}}$. On the other hand, we also construct the BFS tree.
\item Implement the Misra and Gries edge coloring algorithm $^{\cite{misra1992constructive}}$
\end{itemize}
